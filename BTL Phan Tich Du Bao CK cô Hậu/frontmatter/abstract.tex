% \abstractpage{
% \blindtext

% \blindtext

% % It is suggested to have the abstract in both language (Vietnamese and English).
% \newpage
% \begin{center}
%     \vspace*{1pt}
% \Large \textcolor{Crimson}{\textit{This is title of the thesis in Vietnamese}} \normalsize\\
% \vspace*{15pt}
% 	{\bf Tóm tắt đồ án} \rm
% \end{center}

% \blindtext

% \blindtext
% }

\chapter*{Tóm tắt}

\cite{chen2023analysis}Bài toán phân tích và dự đoán giá Bitcoin trong bối cảnh thị trường tiền mã hóa đầy biến động đã trở thành một thách thức quan trọng đối với các nhà đầu tư và các công ty tài chính. Bài báo cáo tiểu luận của chúng tôi sẽ trình bày, phân tích, mở rộng và đánh giá một bài báo nghiên cứu tiềm năng tập trung vào phương pháp sử dụng ứng dụng học máy. Mục tiêu của nghiên cứu này là phát triển một mô hình thuật toán với độ chính xác cao trong việc dự đoán giá Bitcoin vào ngày tiếp theo, thông qua việc áp dụng các phương pháp hồi quy rừng ngẫu nhiên (Random Forest) và LSTM, đồng thời phân tích những yếu tố ảnh hưởng đến giá của Bitcoin.

Nghiên cứu tập trung vào việc so sánh hai mô hình học máy là LSTM và Random Forest để kiểm chứng khả năng dự đoán giá Bitcoin, nhằm trả lời câu hỏi liệu các mô hình này có thể cung cấp dự báo chính xác trong bối cảnh biến động của thị trường tiền mã hóa hay không. Các nghiên cứu trước đây đã chỉ ra rằng Bitcoin là một tài sản có độ biến động cao, khiến các phương pháp dự đoán truyền thống gặp nhiều thách thức trong việc nhận diện xu hướng giá. Việc áp dụng các kỹ thuật học máy được coi là cách tiếp cận hiện đại, nhằm cải thiện khả năng dự đoán và cung cấp góc nhìn sâu sắc hơn về các yếu tố ảnh hưởng đến giá Bitcoin.

Trong nghiên cứu này, dữ liệu được thu thập từ nhiều nguồn khác nhau, bao gồm giá Bitcoin, các chỉ số thị trường tài chính, giá hàng hóa, và mức độ quan tâm của công chúng, từ năm 2015 đến năm 2022. Phương pháp nghiên cứu bao gồm việc áp dụng mô hình LSTM và Random Forest, hai công cụ phổ biến trong phân tích chuỗi thời gian và dự báo.

Kết quả nghiên cứu chỉ ra rằng cả hai mô hình đều có khả năng dự đoán giá Bitcoin tương đối tốt, tuy nhiên, mô hình Random Forest cho thấy hiệu suất dự đoán vượt trội hơn so với LSTM. Điều này khẳng định tính hiệu quả của Random Forest trong việc xử lý dữ liệu phức tạp và đa chiều từ thị trường tiền mã hóa.