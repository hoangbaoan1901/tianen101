\chapter*{Mở đầu}
\addcontentsline{toc}{chapter}{Mở đầu}

Trong những năm gần đây, xu hướng chuyển đổi số và sự bùng nổ của kỷ nguyên dữ liệu lớn đã mở ra nhiều cơ hội cho các công ty và tổ chức trong việc xử lý, giải quyết các bài toán kinh doanh, đặc biệt là trong lĩnh vực tài chính và đầu tư. Thị trường tiền điện tử cũng không nằm ngoài xu thế này khi ngày càng thu hút sự quan tâm mạnh mẽ của các nhà đầu tư, nhà nghiên cứu và các cơ quan quản lý. Bitcoin, đồng tiền điện tử đầu tiên được giới thiệu bởi Satoshi Nakamoto vào năm 2008, đã trở thành biểu tượng của cuộc cách mạng công nghệ tài chính nhờ vào khả năng hoạt động phi tập trung, bảo mật cao và khả năng lưu trữ giá trị. Tuy nhiên, tính biến động mạnh mẽ của Bitcoin đã đặt ra một thách thức lớn trong việc dự đoán giá trị của nó, đồng thời tạo nên nhu cầu nghiên cứu nhằm giảm thiểu rủi ro và tối ưu hóa chiến lược đầu tư cho các bên liên quan.

Việc dự đoán giá Bitcoin trở nên quan trọng khi giá trị của đồng tiền này liên tục trải qua những giai đoạn tăng giảm đột ngột, đặc biệt trong các giai đoạn bùng nổ giá vào năm 2017 và 2021. Tính biến động cao của Bitcoin được thể hiện rõ qua độ lệch chuẩn của tỷ suất lợi nhuận hàng ngày lên tới 3,85\% 
trong khoảng thời gian từ năm 2015 đến 2022, cao hơn nhiều so với vàng hay chỉ số S\&P500. Điều này đặt ra câu hỏi làm thế nào để dự đoán giá Bitcoin một cách hiệu quả nhằm giúp các nhà đầu tư giảm thiểu rủi ro và tận dụng cơ hội. Trong bối cảnh đó, các phương pháp học máy đã nổi lên như một hướng tiếp cận tiềm năng cho bài toán dự đoán giá Bitcoin.

Bài báo "Analysis of Bitcoin Price Prediction Using Machine Learning" của Junwei Chen (2023) đã nghiên cứu việc áp dụng các mô hình học máy - hồi quy rừng ngẫu nhiên (Random Forest Regression) và mạng thần kinh LSTM (Long Short-Term Memory) - để dự đoán giá Bitcoin vào ngày tiếp theo. Hai mô hình này được lựa chọn nhờ vào khả năng phân tích dữ liệu thời gian, trong đó LSTM là mô hình phổ biến trong việc xử lý chuỗi thời gian với tính phụ thuộc, còn hồi quy rừng ngẫu nhiên được đánh giá cao bởi khả năng giải thích các yếu tố ảnh hưởng. Nghiên cứu không chỉ tìm hiểu mô hình nào có độ chính xác cao hơn mà còn đánh giá những yếu tố ảnh hưởng chính đến biến động giá Bitcoin trong các giai đoạn khác nhau.

