% \begin{savequote}[75mm] 
% Nulla facilisi. In vel sem. Morbi id urna in diam dignissim feugiat. Proin molestie tortor eu velit. Aliquam erat volutpat. Nullam ultrices, diam tempus vulputate egestas, eros pede varius leo.
% \qauthor{Quoteauthor Lastname} 
% \end{savequote}

\chapter{Giới thiệu}

Bài báo được phân tích trong bài có tên là “Analysis of Bitcoin Price Prediction Using Machine Learning” được đăng trên tạp chí “Journal of Risk and Financial Management” xuất bản năm 2023 của tác giả Junwei Chen. Bài báo tập trung vào việc dự đoán giá Bitcoin dựa trên việc sử dụng hai mô hình LSTM và Random Forest, sử dụng bộ dữ liệu từ nhiều nguồn khác nhau nhằm nâng cao độ chính xác của dự đoán giá Bitcoin trong tương lai.

\section{Giới thiệu bài toán}

Bitcoin là một loại tiền kỹ thuật số phi tập trung sử dụng mã hóa để đảm bảo an toàn và không bị kiểm soát bởi bất kỳ chính phủ hay tổ chức tài chính nào. Nó được tạo ra vào năm 2008 bởi một cá nhân hoặc nhóm cá nhân sử dụng bút danh Satoshi Nakamoto (2008) với bài báo có tiêu đề \textit{Bitcoin: A Peer-to-Peer (P2P) Electronic Cash System}. Các giao dịch bằng Bitcoin được ghi lại trên một sổ cái công khai gọi là blockchain, cho phép mọi người xem lịch sử của một Bitcoin cụ thể. Tính phi tập trung của Bitcoin cho phép nó hoạt động độc lập với các ngân hàng trung ương và có thể được chuyển ngay lập tức trên toàn cầu. Nó đã trở nên phổ biến như một phương tiện trao đổi và lưu trữ giá trị (Baur và Dimpfl 2021). Trong 10 năm qua, sau nhiều biến động, Bitcoin đã vượt mốc 68.000 USD mỗi đồng vào tháng 11 năm 2021, và tổng giá trị hiện tại từng vượt qua 1,2 nghìn tỷ USD.

Ngày nay, với sự phát triển nhanh chóng và biến động khó lường của thị trường tiền mã hóa, việc dự đoán chính xác giá của các loại tiền mã hóa, đặc biệt là Bitcoin, đã trở thành một bài toán quan trọng đối với các nhà đầu tư và các công ty tài chính. Bitcoin, được biết đến như là một đồng tiền điện tử phi tập trung, đã trải qua nhiều giai đoạn tăng giảm giá mạnh mẽ, điều này làm cho việc nắm bắt xu hướng giá của nó trở thành một thách thức lớn. Chính vì tính chất biến động này, việc xây dựng một mô hình có khả năng dự đoán chính xác giá của Bitcoin trong tương lai gần là rất cần thiết, giúp các nhà đầu tư có thể đưa ra quyết định kịp thời và tối ưu hóa lợi nhuận.

Để có thể dự đoán xu hướng giá của Bitcoin, các phương pháp dựa trên trí tuệ nhân tạo và học máy đã được áp dụng rộng rãi. Những mô hình này có khả năng phân tích dữ liệu lịch sử, bao gồm giá của Bitcoin, giá của các loại tài sản khác như dầu thô, chứng khoán, và các yếu tố ảnh hưởng khác, nhằm trích xuất những thông tin có giá trị và xây dựng các thuật toán dự báo hiệu quả. Dự đoán chính xác giá Bitcoin không chỉ giúp giảm thiểu rủi ro cho nhà đầu tư mà còn góp phần quan trọng trong việc ổn định thị trường tiền mã hóa, nơi mà sự thay đổi giá có thể tạo ra những ảnh hưởng lớn đối với nền kinh tế toàn cầu.

Cùng với xu hướng ứng dụng các thuật toán học máy ngày càng nhiều trong tài chính, bài toán dự đoán giá Bitcoin đang ngày càng thu hút sự quan tâm không chỉ của các nhà đầu tư mà còn của giới nghiên cứu. Lượng dữ liệu khổng lồ và tính chất thời gian thực của thị trường Bitcoin vừa là cơ hội vừa là thách thức đối với các nhà nghiên cứu trong việc trích xuất tri thức và xây dựng các mô hình dự đoán hiệu quả. Việc sử dụng các phương pháp như hồi quy rừng ngẫu nhiên (Random Forest Regression) và mô hình LSTM giúp so sánh hiệu suất của các thuật toán trong việc dự đoán giá và xác định các yếu tố có ảnh hưởng đến giá trị của Bitcoin qua thời gian.

Một trong những ứng dụng quan trọng của bài toán dự đoán giá Bitcoin là giúp các nhà đầu tư đưa ra các chiến lược giao dịch phù hợp, từ đó tối ưu hóa lợi nhuận và giảm thiểu thiệt hại do những biến động không lường trước của thị trường. Điều này đặc biệt quan trọng trong bối cảnh thị trường tiền mã hóa luôn có sự thay đổi mạnh mẽ và không ổn định, yêu cầu sự chuẩn bị và phản ứng nhanh chóng từ phía các nhà đầu tư.

\section{Phát biểu bài toán}
Bài toán dự đoán giá Bitcoin  được mô tả như sau: Đầu vào của bài toán là các thông tin liên quan đến thị trường và Bitcoin, bao gồm giá Bitcoin trong quá khứ, các chỉ số thị trường tài chính, và giá của các loại tài sản khác. Nhiệm vụ của bài toán là dự đoán giá của Bitcoin vào ngày tiếp theo. Bài toán có thể được định nghĩa như sau:

\begin{itemize}
    \item \textbf{Đầu vào}: 
    \begin{itemize}
        \item $X = \{x_1, x_2, \dots, x_n\}$: trong đó $X$ là tập dữ liệu bao gồm các đặc trưng về giá Bitcoin trong quá khứ, giá của các tài sản khác (như dầu thô, vàng, ETH), và các chỉ số thị trường tài chính (như NASDAQ, S\&P500).
    \end{itemize}
    \item \textbf{Đầu ra}: 
    \begin{itemize}
        \item $Y$: là giá trị dự đoán của Bitcoin vào ngày tiếp theo, được biểu diễn dưới dạng một số thực.
    \end{itemize}
\end{itemize}

\section{Khó Khăn và Thách Thức}

Dự báo giá Bitcoin là một bài toán phức tạp do tính biến động cao của giá cũng như sự ảnh hưởng từ các yếu tố ngoại sinh. Khó khăn thứ nhất, tính biến động cao và khó lường của giá Bitcoin là một trong những thách thức lớn nhất đối với các mô hình dự đoán. Bitcoin có tính thanh khoản cao và rất nhạy cảm với các yếu tố bên ngoài, bao gồm sự thay đổi chính sách của các quốc gia, các sự kiện kinh tế toàn cầu, hoặc phát ngôn của những nhân vật có tầm ảnh hưởng. Những sự kiện này có thể gây ra những biến động mạnh mẽ và đột ngột trong giá Bitcoin, khiến cho việc xây dựng một mô hình dự đoán ổn định và chính xác trở nên rất khó khăn.

Thứ hai, hạn chế của dữ liệu huấn luyện cũng là một thách thức. Mặc dù mô hình học máy có thể dựa vào dữ liệu quá khứ để dự đoán tương lai, nhưng khi xuất hiện những biến động lớn hoặc những tình huống bất ngờ, việc dự đoán trở nên thiếu chính xác. Điều này đặc biệt đúng trong giai đoạn Bitcoin phá vỡ các mức giá cao kỷ lục, khi dữ liệu quá khứ không đủ để phản ánh mức giá mới. Hơn nữa, việc thiếu hụt hoặc không đồng nhất về dữ liệu của thị trường tiền mã hóa cũng là một trở ngại trong việc cải thiện hiệu quả của các mô hình dự đoán.

Thứ ba, việc lựa chọn và xử lý các biến số giải thích. Bài toán dự đoán giá Bitcoin đòi hỏi phải xác định được những biến số nào thực sự ảnh hưởng đến giá. Sự phức tạp của thị trường tiền mã hóa và mối liên hệ giữa Bitcoin với các tài sản khác như chỉ số chứng khoán, giá dầu, và các đồng tiền mã hóa khác khiến cho việc lựa chọn biến số và tối ưu hóa mô hình trở nên khó khăn. Việc lựa chọn sai biến số có thể làm giảm độ chính xác của mô hình.

Khó khẳn cuối cùng, hiện tượng quá khớp (overfitting) là một rủi ro khi sử dụng các mô hình học máy, đặc biệt là mô hình học sâu như LSTM. Khi mô hình học quá kỹ vào dữ liệu huấn luyện, nó có thể không hoạt động hiệu quả khi áp dụng vào dữ liệu mới. Việc cân bằng giữa độ phức tạp của mô hình và khả năng tổng quát hóa là một thách thức lớn, đòi hỏi phải tối ưu hóa cẩn thận các tham số của mô hình để tránh hiện tượng này. Bài báo gốc cũng nhấn mạnh rằng việc tối ưu hóa các tham số như hệ số dropout là cần thiết để đạt hiệu quả tốt và tránh hiện tượng quá khớp.

\section{Công Trình Liên Quan}

Các nghiên cứu trước đây đã khám phá nhiều phương pháp dự đoán giá Bitcoin, sử dụng cả học sâu và học máy. Aggarwal và cộng sự (2019) đã nghiên cứu khả năng dự đoán giá Bitcoin từ giá vàng thông qua ba thuật toán học sâu: Mạng Nơ-ron Tích chập (CNN), Mạng Bộ nhớ Dài hạn (LSTM), và Đơn vị Bộ nhớ Cổng (GRU). Kết quả chỉ ra rằng mô hình LSTM đạt độ chính xác cao nhất trong ba mô hình, mặc dù tồn tại sự chênh lệch giữa giá Bitcoin dự đoán và giá thực tế khi chỉ sử dụng giá vàng làm yếu tố dự đoán.

Liu và cộng sự (2021) \cite{6} đã mở rộng phạm vi của các biến giải thích, bao gồm dữ liệu từ thị trường tiền điện tử, chỉ số kinh tế vĩ mô (ví dụ như chỉ số chứng khoán, giá dầu thô, tỷ giá hối đoái) và chỉ số tìm kiếm, với tổng cộng 40 biến giải thích. Kết quả cho thấy Stacked Denoising Autoencoder (SDAE) vượt trội hơn các mô hình như Mạng Nơ-ron Lan Truyền Ngược (BPNN), Phân Tích Thành Phần Chính - Hồi quy Hỗ trợ Vector (PCA-SVR), và Hồi quy Hỗ trợ Vector (SVR).

Nhiều nghiên cứu đã phát hiện rằng các phương pháp học máy cung cấp độ chính xác cao hơn cho dự đoán giá Bitcoin so với các mô hình chuỗi thời gian truyền thống như ARIMA (McNally và cộng sự, 2018 \cite{8}; Shin và cộng sự, 2021 \cite{9}; Chen và cộng sự, 2020a \cite{10}; Akyildirim và cộng sự, 2021 \cite{11}). Đặc biệt, mô hình LSTM đã được nghiên cứu rộng rãi. Phaladisailoed và Numnonda (2018) \cite{12} đã áp dụng bốn thuật toán học sâu (hồi quy Theil-Sen, hồi quy Huber, LSTM và GRU) để dự đoán giá Bitcoin, với LSTM đạt độ chính xác cao nhất (52,78\%). Tương tự, Tandon và cộng sự (2019) cho thấy việc áp dụng kiểm tra chéo mười lần vào quá trình huấn luyện LSTM có thể cải thiện độ chính xác lên 14,7\%.

Aggarwal và cộng sự (2019) cũng đưa giá vàng vào làm một biến giải thích bổ sung cho giá Bitcoin, và mô hình LSTM đạt được RMSE (Root Mean Square Error) là 47,91, vượt trội hơn CNN và GRU. McNally và cộng sự (2018) \cite{8} bổ sung các biến như độ khó và tỷ lệ băm của Bitcoin, và nhận thấy LSTM có độ chính xác dự đoán tốt hơn (52,78\%) so với Mạng Nơ-ron Hồi quy (RNN) và ARIMA. Chen và cộng sự (2020a) \cite{10} đã sử dụng LSTM, SVR, Hệ Thống Suy Luận Thần Kinh - Mờ Thích Nghi (ANFIS), và ARIMA, thêm các thuộc tính cụ thể của Bitcoin, các biến công khai (ví dụ như Google Trends và dữ liệu Twitter), và các biến kinh tế. Trong tất cả bốn giai đoạn mẫu, LSTM đều vượt trội hơn ba mô hình còn lại. Bên cạnh đó, Livieris và cộng sự (2020) đã giới thiệu một khung tiền xử lý giúp cải thiện hiệu suất dự đoán của LSTM bằng cách biến đổi dữ liệu dựa trên sự khác biệt đầu tiên hoặc lợi nhuận.

Ngoài việc dự đoán giá Bitcoin, LSTM còn được sử dụng để dự đoán các loại tiền điện tử khác. Sebastião và Godinho (2021), Saadah và Whafa (2020), và Derbentsev và cộng sự (2020) đã áp dụng LSTM để dự đoán giá của Ether, với Politis và cộng sự (2021) báo cáo độ chính xác là 84,2\%. Livieris và cộng sự (2021) sử dụng mô hình kết hợp CNN-LSTM để dự đoán giá của Bitcoin (BTC), Ethereum (ETH), và Ripple (XRP), với độ chính xác lần lượt là 55,03\% cho BTC, 51,51\% cho ETH, và 49,61\% cho XRP.

Trong một số nghiên cứu (McNally và cộng sự, 2018\cite{8}; García-Medina và Duc Huynh, 2021; Chen và cộng sự, 2020a \cite{10}), việc thêm các lớp dropout giữa các lớp LSTM đã được chứng minh giúp giảm thiểu hiện tượng quá khớp. Việc lựa chọn hệ số dropout (0,1, 0,3, và 0,5) khác nhau giữa các nghiên cứu, và điều này có thể ảnh hưởng đến khả năng tổng quát của mô hình.

Các biến giải thích đóng vai trò quan trọng trong việc nâng cao độ chính xác của dự đoán giá Bitcoin. Ngoài các yếu tố kinh tế vĩ mô, nghiên cứu của Jagannath và cộng sự (2021) tập trung vào các biến cốt lõi của blockchain Bitcoin, bao gồm người dùng, thợ đào và sàn giao dịch. Nghiên cứu của Jaquart và cộng sự (2021) và Mudassir và cộng sự (2020) xác nhận rằng các chỉ báo kỹ thuật có ích trong việc dự đoán giá Bitcoin. García-Medina và Duc Huynh (2021) cũng nghiên cứu tác động của mạng xã hội, bao gồm các bình luận của Elon Musk và Donald Trump, nhưng không tìm thấy sức mạnh giải thích đáng kể từ những yếu tố này trong năm 2020.

Về lựa chọn đơn vị thời gian để dự đoán giá, hầu hết các nghiên cứu sử dụng dữ liệu theo ngày hoặc phút. Lamothe-Fernández và cộng sự (2020) tiến hành dự đoán theo quý bằng các mô hình như Mạng Biến Động Hồi Quy Chồng Chất (DSVR), Chuyển Tiếp Phi Tuyến Sâu (DNDT), và Mạng Nơ-ron Tích Chập Hồi Quy Sâu (DRCNN), đạt được độ chính xác trên 60\%. Tuy nhiên, độ chính xác cao này có thể bị ảnh hưởng bởi xu hướng tăng chung của Bitcoin trong giai đoạn nghiên cứu (2011-2019). Shin và cộng sự (2021) nhận thấy rằng mô hình theo ngày và phút có độ chính xác dự đoán tương đương nhau và tốt hơn so với mô hình theo giờ.

Nhìn chung, các nghiên cứu liên quan đã chứng minh rằng các phương pháp học máy, đặc biệt là LSTM, mang lại độ chính xác cao hơn trong việc dự đoán giá Bitcoin và các loại tiền điện tử khác so với các phương pháp thống kê truyền thống. Sự kết hợp giữa các biến giải thích phù hợp và kỹ thuật tiền xử lý đóng vai trò quan trọng trong việc nâng cao hiệu suất của các mô hình này.